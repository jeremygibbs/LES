 \documentclass{beamer}[10]

\usepackage{graphicx}
\usepackage{xcolor}
\usepackage{tabto}
%\usepackage{beamerthemesplit}
\usepackage{tikz}
\usepackage{cancel}
\usepackage{verbatim}
\usepackage{fancybox}
\usepackage{enumerate}
\usepackage{amsmath,amssymb,amsthm,textcomp,mathtools}
\usepackage[super]{nth}
\usepackage[amssymb]{SIunits}
\usepackage{booktabs}
\usepackage{cancel}
\usepackage{bm}
\usepackage[utf8]{inputenc}
\usepackage{tabularx}
\usepackage{ragged2e}
\newcolumntype{Y}{ >{\RaggedRight\arraybackslash}X}
\usetikzlibrary{arrows,shapes}
\newcommand\T{\rule{0pt}{2.6ex}}
\newcommand\B{\rule[-1.2ex]{0pt}{0pt}}
\definecolor{UUcrimson}{RGB}{204,0,0}
\mode<presentation>
{ \usetheme{default}
  \usecolortheme[named=UUcrimson]{structure}
  \useinnertheme{circles}
  \setbeamercovered{transparent}
  \setbeamertemplate{blocks}[rounded]
  \usefonttheme[onlymath]{serif}
  \setbeamertemplate{navigation symbols}{}
  \setbeamertemplate{footline}[page number]
  \setbeamertemplate{navigation symbols}{}
  \setbeamercolor{section in toc}{fg=black,bg=white}
  \setbeamercolor{alerted text}{fg=UUcrimson!80!gray}
  \setbeamercolor*{palette primary}{fg=white,bg=UUcrimson}
  \setbeamercolor*{palette secondary}{fg=UUcrimson!70!black,bg=gray!15!white}
  \setbeamercolor*{palette tertiary}{bg=UUcrimson!80!black,fg=gray!10!white}
  \setbeamercolor*{palette quaternary}{fg=UUcrimson,bg=gray!5!white}
  \setbeamercolor*{palette sidebar primary}{fg=UUcrimson!10!black}
  \setbeamercolor*{palette sidebar secondary}{fg=white}
  \setbeamercolor*{palette sidebar tertiary}{fg=UUcrimson!50!black}
  \setbeamercolor*{palette sidebar quaternary}{fg=gray!10!white}
  \setbeamercolor{titlelike}{parent=palette primary,fg=white}
  \setbeamercolor{frametitle}{bg=UUcrimson}
  \setbeamercolor{frametitle right}{bg=UUcrimson}
  \setbeamercolor*{separation line}{}
  \setbeamercolor*{fine separation line}{}
}

\usetikzlibrary{backgrounds}
\makeatletter
\tikzstyle{every picture}+=[remember picture]
\tikzset{%
  fancy quotes/.style={
    text width=\fq@width pt,
    align=justify,
    inner sep=1em,
    anchor=north west,
    minimum width=\linewidth,
    font=\itshape
  },
  fancy quotes width/.initial={.8\linewidth},
  fancy quotes marks/.style={
    scale=8,
    text=white,
    inner sep=0pt,
  },
  fancy quotes opening/.style={
    fancy quotes marks,
  },
  fancy quotes closing/.style={
    fancy quotes marks,
  },
  fancy quotes background/.style={
    show background rectangle,
    inner frame xsep=0pt,
    background rectangle/.style={
      fill=gray!25,
      rounded corners,
    },
  }
}
\newenvironment{fancyquotes}[1][]{%
\noindent
\tikzpicture[fancy quotes background]
\node[fancy quotes opening,anchor=north west] (fq@ul) at (0,0) {``};
\tikz@scan@one@point\pgfutil@firstofone(fq@ul.east)
\pgfmathsetmacro{\fq@width}{\linewidth - 2*\pgf@x}
\node[fancy quotes,#1] (fq@txt) at (fq@ul.north west) \bgroup}
{\egroup;
\node[overlay,fancy quotes closing,anchor=east] at (fq@txt.south east) {''};
\endtikzpicture}
\makeatother

\usepackage{scalerel}[2014/03/10]
\usepackage{stackengine}
\usepackage{empheq}
\newcommand*\widefbox[1]{\fbox{\hspace{0.5em}#1\hspace{0.5em}}}

\newcommand\reallywidetilde[1]{\ThisStyle{%
  \setbox0=\hbox{$\SavedStyle#1$}%
  \stackengine{-.1\LMpt}{$\SavedStyle#1$}{%
    \stretchto{\scaleto{\SavedStyle\mkern.2mu\sim}{.5467\wd0}}{.4\ht0}%
%    .2mu is the kern imbalance when clipping white space
%    .5467++++ is \ht/[kerned \wd] aspect ratio for \sim glyph
  }{O}{c}{F}{T}{S}%
}}
\usepackage{media9}

\logo{\includegraphics[width=0.75cm]{logo.jpg}}
\author[Gibbs]{Dr. Jeremy A. Gibbs}
\institute{Department of Mechanical Engineering\\University of Utah}
\date{Fall 2016}
\title{LES of Turbulent Flows: Lecture 17}
\begin{document}

%----------------------------------------------------------------------------------------
%	TITLE & TOC SLIDES
%----------------------------------------------------------------------------------------

\begin{frame} 
  \titlepage
\end{frame}

%------------------------------------------------

\begin{frame}
\frametitle{Overview}
\tableofcontents
\end{frame}

%------------------------------------------------
\section{Stochastic Burgers Equation} %
%------------------------------------------------
\begin{frame}{Stochastic Burgers Equation}
\begin{itemize}
	\item Originally conceived by Dutch scientist J.M. Burgers in the 1930s
	\item One of the first attempts to arrive at the statistical theory of turbulent fluid motion
	$$\frac{\partial u}{\partial t} + u\frac{\partial u}{\partial x} = \nu \frac{\partial^2 u}{\partial x^2}$$
	This represents a very simplified model that describes the interaction of non-linear inertial terms and dissipation in the motion of a fluid
\end{itemize}
\end{frame}
%------------------------------------------------
\begin{frame}{Stochastic Burgers Equation}
The original Burgers equation shares a lot in common with the N-S equations
\begin{itemize}
	\item advective nonlinearity
	\item diffusion (can compute Re)
	\item invariance and conservation laws, such as translation in space and time
\end{itemize}
\end{frame}

%------------------------------------------------
\begin{frame}{Stochastic Burgers Equation}
\begin{itemize}
	\item There were downsides discovered
	\item The equation can be integrated explicitly -- meaning that it does not share the N-S equations sensitivity to small changes in initial conditions in presence of boundaries, forcing, and at high Re
	\item Thus Burgers equation is not an ideal model for the chaotic nature of turbulence
	\item It was also found that shock waves form in the limit of vanishing viscosity
	\end{itemize}
\end{frame}

%------------------------------------------------
\begin{frame}{Stochastic Burgers Equation}
\begin{itemize}
	\item The use of Burgers with a forcing term has been popular because the original model is an incomplete description of a turbulent system
	\item The forcing term can account for the neglected effects
	\item For instance, one may perturb the system with a stochastic process that is stationary in time and space (this preserves translational invariance)
	\item One example is white noise - which preserves Galilean invariance 
	\end{itemize}
\end{frame}

%------------------------------------------------

\begin{frame}{Stochastic Burgers Equation}
\begin{itemize}
	\item Project \#1 is based on a useful model of the Navier-Stokes equations: the 1D Stochastic Burgers Equation
	$$\frac{\partial u}{\partial t} + u\frac{\partial u}{\partial x} = \nu \frac{\partial^2 u}{\partial x^2} + \eta(x,t)$$
	\item This analog has been found to be a useful one for the study of turbulent like nonlinear systems (e.g., Basu, 2009 and references contained within).
	\item Although 1D, this equation has some of the most important characteristics of a turbulent flow -- making it a good model case
\end{itemize}

\end{frame}

%------------------------------------------------
\begin{frame}{Stochastic Burgers Equabasution}
\begin{itemize}
	\item In short, the Burgers system has been popular because it allowed one to gain insight into turbulence structure before having to generalize for the fully-3D case
	\item It shares many characteristics of 3D turbulence, such as nonlinearity, energy spectrum, intermittent energy dissipation
	\item The system is also super-cheap computationally
	\end{itemize}
\end{frame}

%------------------------------------------------
\begin{frame}{Stochastic Burgers Equation}
$$\frac{\partial u}{\partial t} + u\frac{\partial u}{\partial x} = \nu \frac{\partial^2 u}{\partial x^2} + \eta(x,t)$$
\begin{itemize}
	\item In the above equation, the ``new'' term is $\eta$ -- which is called the stochastic term
	\item $\eta(x,t)$ should be white noise in time, but spatially correlated
\end{itemize}

\end{frame}

%------------------------------------------------
\begin{frame}{Stochastic Burgers Equation}
\begin{itemize}
	\item Here we use
	$$\eta(x,t) = \sqrt{\frac{2D_0}{\Delta t}} \mathfrak{F}^{-1} \left\lbrace|k|^{\beta/2}\hat{f}(k)\right\rbrace$$
	where
	\begin{align*}
	D_0 =& \text{ noise amplitude}\\
	\Delta t = &\text{ time step}\\
	\mathfrak{F}^{-1} =& \text{ inverse Fourier transform}\\
	f =&\text{ Gaussian random noise with mean = 0 and}\\ 
	&\text{ standard deviation = $\sqrt{N}$ (where $N$ is \# of points)}\\
	\beta =& \text{ spectral slope of the noise (taken as $-3/4$ here)} 
	\end{align*}
\end{itemize}

\end{frame}

%------------------------------------------------
\begin{frame}{Stochastic Burgers Equation}
\begin{itemize}
	\item Many solutions exist for stochastic Burgers equation
	\item Here we follow Basu (2009) -- Fourier collocation
	\item Basically, use Fourier methods but advance time in real space (compare to Galerkin)
	\item To do this, the main numerical methods we need to know are how to calculate \underline{derivatives} and how to \underline{advance time}
\end{itemize}

\end{frame}

%------------------------------------------------
\begin{frame}{Fourier Derivatives}

\begin{itemize}
	\item Mathematically the discrete Fourier transform pair is (see also Lecture 3 as a supplement)
	\begin{align}
		\label{backward}
		f(x_j) &= \sum^{N/2 -1}_{m = -N/2} \hat f(k_m) e^{i k_m x_j} \rightarrow \text{backward transform}\\
		\label{forward}
		\underbrace{\hat f(k_m)}_{\substack{\text{Fourier}\\\text{coeffs}}} &= \frac{1}{N} \sum^N_{j=1} f(x_j) e^{-i k_m x_j} \rightarrow \text{forward transform}
	\end{align}
	where
	$$k_m = \frac{2\pi m}{N\Delta x} \rightarrow \text{wave number (wave period)}$$
	recall
	$$e^{-ik_mx_j} = \cos (k_mx_j) + i\sin (k_m x_j)$$
\end{itemize}

\end{frame}

%------------------------------------------------
\begin{frame}{Fourier Derivatives}
\begin{itemize}
	\item How is this used numerically to calculate a derivative?
	\item  A Fourier series can be used to interpolate $f(x_j)$ at any point $x$ in the flow and at any time $t$
	\item  If we differentiate the Fourier representation of $f(x_j)$ (Eq. ~\ref{backward}) with respect to $x$
	\begin{align*}
	\frac{\partial f}{\partial x} &= \frac{\partial}{\partial x} \left[ \sum^{N/2 -1}_{m = -N/2} \hat f(k_m) e^{i k_m x_j} \right]\\
	&= \sum^{N/2 -1}_{m = -N/2} \hat f(k_m) \frac{\partial e^{i k_m x_j}}{\partial x}\\
	\Aboxed{&=\sum^{N/2 -1}_{m = -N/2} i k_m \hat f(k_m) e^{i k_m x_j}}
	\end{align*}
	\end{itemize}
\end{frame}

%------------------------------------------------
\begin{frame}{Fourier Derivatives}
\begin{itemize}
	\item If we compare this to Eq.~(\ref{backward}), we notice that we have
	\begin{align*}
		\frac{\partial f}{\partial x} = g &= \sum^{N/2 -1}_{m = -N/2}\underbrace{i k_m \hat f(k_m)}_{\hat g(k_m)} e^{i k_m x_j}\\
		\Aboxed{&=\sum^{N/2 -1}_{m = -N/2} \hat g (k_m) e^{i k_m x_j}}
	\end{align*}
\end{itemize}
\end{frame}

%------------------------------------------------
\begin{frame}{Fourier Derivatives}
Procedurally, we can use this to find $\left.\frac{\partial f}{\partial x_j}\right|_j$ given $f(x_j)$ as follows:
\begin{itemize}
	\item calculate $\hat f(k_m)$ by the forward transform (Eq.~\ref{forward})
	\item multiply by $i k_m$ to get $\hat g (k_m)$, and then
	\item perform a backward (inverse) transform (Eq.~\ref{backward}) to get $\left.\frac{\partial f}{\partial x_j}\right|_j$
\end{itemize}
The method easily generalizes to any order derivative

\end{frame}

%------------------------------------------------
\begin{frame}{Fourier Derivatives}
Although Fourier methods are quite attractive due to their high accuracy and near-exact representation of derivatives, they have a few important limitations
\begin{itemize}
	\item $f(x_j)$ must be continuously differentiable
	\item $f(x_j)$ must be periodic
	\item grid spacing must be uniform
\end{itemize}

\end{frame}

%------------------------------------------------
\begin{frame}{Time Advancement}
\begin{itemize}
	\item Time advancement in this code is accomplished using a \nth{2}-order Adams-Bashforth scheme
	\item This is a basic extension of the Euler method -- multipoint (in time), with the idea to fit a polynomial of desired order (\textit{e.g.}, \nth{2}) through 3 points in time to get
	$$\phi^{n+1} = \phi^n + \frac{\Delta t}{2} \left[ 3f(t^n,\phi^n) - f(t^{n-1}, \phi^{n-1})\right]$$
	\item For specifics on how these things are implemented, see the Matlab code on the course website or on Canvas
\end{itemize}

\end{frame}

%------------------------------------------------



\end{document}

